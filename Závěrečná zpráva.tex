\documentclass{article}

\usepackage[utf8]{inputenc}
\usepackage{chngcntr}
\usepackage{fullpage}
\usepackage{tikz}
\usepackage{amsmath}
\usepackage{pgfplots}
\usepackage{listingsutf8}
\usepackage{graphicx}
\usepackage[main=czech]{babel}
\usepackage[utf8]{inputenc}
\usepackage{graphicx}
\usepackage{amsfonts}
\usepackage{hyperref}

\title{Závěrečná zpráva Hatalom}
\author{Patrik Novotný, Jakub Suchánek, Lukáš Vőlfl, Jan Janoušek, Jakub Zeman, Martin Vítek}
\date{2018}
\begin{document}

\maketitle
\tableofcontents
\section{Náš tým}
\item Patrik Novotný – výkonný vedoucí
\item Jakub Suchánek – technický vedoucí
\item Lukáš Vőlfl – konstruktér, dronař
\item Jan Janoušek – PR a HR odborník
\item Jakub Zeman – konstruktér, bezpečnost práce
\item Martin Vítek – programátor
\item Mimo svoje pozice se dělíme do týmů dle daného úkolu - hlavně programátorský (Jakub Suchánek, Patrik Novotný a Martin Vítek) a konstrukční (Jakub Zeman, Lukáš Vőlfl a Jakub Suchánek), Jan Janoušek obstarává provozní činnosti.
\section{Letové pozice}
\item Speciálně pak máme vyčleněny pozice pro let (v současné době se prakticky týká pouze testů komponent). Pro neexistenci českých ekvivalentů užíváme anglických názvů:
\item Mission Manager – Patrik Novotný
\item Payload Manager – Jakub Suchánek
\item Flight Director – Lukáš Vőlfl
\item Safety Officer– Jakub Zeman
\section{Průběh prací}
\section{Příběh}
\item Za každou událostí v dějinách stojí příběh, protože právě příběhy jsou tím, co člověka motivuje k činnosti. Jaký je náš příběh? Proč stavíme meziplanetární sondu? Simulujeme misi, ke které možná jednoho dne dojde. Cílem našeho zájmu je planeta obíhající kolem jiné hvězdy, která se jmenuje Kepler 62, okem byste ji neviděli. Je menší naše Slunce. Z našich zemí se zdá, že obíhá vysoko nad našimi hlavami v souhvězdí Lyry. Je od nás vzdálena skoro 1200 světelných let. 
\item Hvězd je ve vesmíru mnoho, astronomové by vám řekli, že v každé galaxii jich je průměrně řádově 1012 a tolik, že je ve vesmíru galaxií. O to ale nejde. Je jich mnoho. A některé z nich mají i své planety. Kepler je právě takovou hvězdou, obíhá ho dle našich poznatků 5 planet, dvě obzvlášť zajímavé, označované jako „e“ a „f“. 
\item Planeta „e“ leží na vnitřní okraji obyvatelné zóny. Prostě řečeno, předpokládáme-li podmínky pro vznik života, pak je moc teplá, má podobnou smůlu asi jako naše Venuše. Ovšem její sourozenec Kepler 62 f je nadějnější. Nemáme zdaleka všechna data, víme, že v dané oběžné vzdálenosti by planeta měla mít průměrnou povrchovou teplotu kolem -40 stupňů. Ovšem pokud by měla tlustší atmosféru (5krát) tvořenou CO2, což není nic neobvyklého, pak by na většině povrchu planety mohla existovat voda v kapalném skupenství. Při tenčí atmosféře by se voda v kapalném stádiu mohla vyskytovat pouze v některých částech planety.
\item Nyní ovšem k optimističtějším datům, hvězda Kepler 62 je stabilní hvězdou (nízká hmotnost i metalicita vůči Slunci), navíc by měla zářit ještě 2x tak dlouho, jak je starý vesmír. Planeta oběhne svou domovskou hvězdu za 267 dní, hvězda by ji tudíž neměla příliš negativně ovlivňovat. Celý systém je dle odhadů až o 2 miliardy let starší než náš. Naše cílová planeta má také pravděpodobně vlastní měsíc a skloněnou rotační osu, takže se na jejím povrchu mění roční období. Z těchto údajů lze odhadnout, že není lepší kandidát pro nalezení mimozemského života, byť lze předpokládat nižší stupeň vyspělosti vzhledem k tomu, že z planety nepřijímáme vysílání.
\item Co a proč tedy hodláme na planetě uskutečnit?
\begin{itemize}
\item Z orbity získáme snímky planety – v našem případě Google Maps, dle snímků vybereme místo přistání

\item Po průletu atmosférou rušící vysílání zahájíme samotnou misi – příjem signálu budeme možný z výšky kolem 1-2 km nad povrchem, budeme udržovat rychlost klesání 6-12 m/s, abychom nepoškodili aparaturu v sondě ani nenarušili přenosy dat uskutečňované více způsoby
\item Celou dobu letu budeme vysílat obraz, je možné, že sice již nežije na planetě vyspělá civilizace, ale třeba žila a její pozůstatky budou viditelné
\item Budeme měřit údaje nutné pro vznik života – množství CO2 a O2, také tlak a teplotu, dále intenzitu magnetického pole
V poslední fázi mise odebereme malý vzorec horniny a zjistíme, zda nám v agaru nevyroste bakteriální život blízký pozemskému
\end{itemize}
\item Na reálnou misi na Kepler 62 f si budeme muset jistě nějakou dobu počkat, ač se dá očekávat, že to bude jeden z prvních lidských mezihvězdných cílů. Na simulaci mise však čekat nemusíme!
\section{Předletová příprava}

\section{Průběh letu}

\section{Vyhodnocení dat}

\section{Hardwarové vybavení}

\section{Softwarové vybavení}

\section{Vizualizace}
\section{Problémy}
\section{Výhled do budoucna}
 ---- Univerzitní a Evropské kolo ---- popsat!
\section{Rozpočet}
\item V rozpočtu jsou zařazeny naše celkové příjmy a výdaje. Neuvádíme vybavení zapůjčené nebo využité, za ty děkujeme sponzorům níže, ale zde jsou pouze uvedena čistá finanční data. Prováděli jsme nákupy v dolarech a eurech, tyto nákupy jsme převedli dle údajů v ČNB ke dni nákupu.
\section{Sponzoři a podporovatelé}

\section{Publicita}
\item Prvně těm, kteří nás podpořili finančně - generálnímu partneru EPSON, kterým nám daroval roll-up a zajistí nám v případě postupu do evropského kola širokou publicitu - výstup z finále přes PR agenturu a účast pořadu typu Snídaně s Novou s cílem popularizovat náš projekt i celou soutěž.  Dále společnostem Rotorama a PCBWay, které nám poskytly slevy ze svých výrobků. Také děkujeme Klubu přátel Gymnázia Opatov, který nám poskytl bezúročnou půjčku na samotný vývoj CanSatu, kterou jsme částečně vrátili z peněz sponzorů a zbytek nezplacenou část nám odpustil jako dar. Také nám KPGO poskytlo právní záštitu. Další poděkování patří Jiřímu Šefkovi z Ústavu pro nanomateriály, pokročilé technologie a inovace z Liberecké univerzity, který nám vytiskl dvě verze CanSatu (včetně finální) na tamní špičkové 3D tiskárně. Poslední díky patří Jakubovi Rozehnalo, řediteli, Hvězdárny a Planetária Praha, který nám zapůjčil nutné vybavení.
\item Za konzultace patří speciální díky naší paní profesorce Lence Vinklerové, odborníkovi na telekomunikaci Zdeňku Mackovi a členovi týmu loňských a předloňských vítězů národního kola soutěže CanSat Janu Horákovi. Za sjednávání kontaktů děkujeme panu Luboru Basíkovi.
Naše webové stránky: \url{http://www.gosa.cz}\\
\section{Doslov členů}
\begin{itemize}
\item Patrik Novotný:
\item Jakub Suchánek:
\item Lukáš Vőlfl:
\item Jan Janoušek:
\item Jakub Zeman:
\item Martin Vítek:
\end{itemize}
\end{document}

