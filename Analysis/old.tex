\documentclass{aa}
\usepackage[utf8]{inputenc}
\usepackage{chngcntr}
% \usepackage[margin=1in,includeheadfoot,headheight=13.6pt,footskip=2cm]{geometry}
\usepackage{amsmath}
\usepackage{pgfplots}
\usepackage{listingsutf8}
\usepackage{graphicx}
\usepackage[main=czech]{babel}
\usepackage{amsfonts}
\usepackage[colorlinks=true,urlcolor=blue]{hyperref}
\usepackage{eurosym}
\usepackage{chemfig}
\usepackage[version=4]{mhchem}
\usepackage{multicol}
% \usepackage{titlesec}
% \usepackage{float}
% \usepackage{fancyhdr}
% \usepackage{array}
% \usepackage{longtable}
% \usepackage{dirtytalk}
\pdfminorversion=5 
\pdfcompresslevel=9
\pdfobjcompresslevel=2
% \newcolumntype{L}[1]{>{\raggedright\let\newline\\\arraybackslash\hspace{0pt}}m{#1}}
% \newcolumntype{C}[1]{>{\centering\let\newline\\\arraybackslash\hspace{0pt}}m{#1}}
% \newcolumntype{R}[1]{>{\raggedleft\let\newline\\\arraybackslash\hspace{0pt}}m{#1}}
% \newenvironment{MulticolFigure}
%   {\par\medskip\noindent\minipage{\linewidth}}
%   {\endminipage\par\medskip}
\title{Final report Hatalom}
\author{Patrik Novotný, Jakub Suchánek, Lukáš Vőlfl,\\Jan Janoušek, Jakub Zeman, Martin Vítek and teacher Mgr. Lenka Vinklerová}
\date{2018}
% \counterwithout{section}{chapter}
% \newcommand{\sectionbreak}{\clearpage}
\def\tick{\tikz\fill[scale=0.4](0,.35) -- (.25,0) -- (1,.7) -- (.25,.15) -- cycle;}
% \pagestyle{fancy}
% \fancyhf{}
% \fancyhead[RO, LE]{\scshape\nouppercase{\leftmark}}
% \fancyfoot[RO, LE]{\thepage}
%\fancyhf{}
% \fancyhead[C]{Gymnazium Opatov Space Agency - Hatalom 2018}
% \fancyfoot[R]{\includegraphics[width=30pt]{Logo_final.png}}
% \fancyfoot[C]{\thepage}
% \renewcommand{\headrulewidth}{0.7pt}
% \renewcommand{\footrulewidth}{0.7pt}
% \setlength{\parskip}{5pt}
% \begin{document}
% \begin{titlepage}
%     \centering
%     \vfil
%     \includegraphics[width=400pt]{Logo_final.png}
%     \vfill
%     {\bfseries\Large
%         Cansat final paper\\
%         Hatalom ´- Czechia  \\
%         2018\\
%     }
%     \vskip2cm
%     \large Patrik Novotný, Jakub Suchánek\\
%     \vfill
%     \large Teacher Mgr. Lenka Vinklerová\\
%     \vfill
%     \vfill
% \end{titlepage}
% \end{document}
\begin{document} 


   \maketitle
   \author
   \date

% \abstract{}{}{}{}{} 
% 5 {} token are mandatory
 
  \abstract
  % context heading (optional)
  % {} leave it empty if necessary  
   {To investigate the physical nature of the `nuc\-leated instability' of
   proto giant planets, the stability of layers
   in static, radiative gas spheres is analysed on the basis of Baker's
   standard one-zone model.}
  % aims heading (mandatory)
   {It is shown that stability
   depends only upon the equations of state, the opacities and the local
   thermodynamic state in the layer. Stability and instability can
   therefore be expressed in the form of stability equations of state
   which are universal for a given composition.}
  % methods heading (mandatory)
   {The stability equations of state are
   calculated for solar composition and are displayed in the domain
   $-14 \leq \lg \rho / \mathrm{[g\, cm^{-3}]} \leq 0 $,
   $ 8.8 \leq \lg e / \mathrm{[erg\, g^{-1}]} \leq 17.7$. These displays
   may be
   used to determine the one-zone stability of layers in stellar
   or planetary structure models by directly reading off the value of
   the stability equations for the thermodynamic state of these layers,
   specified
   by state quantities as density $\rho$, temperature $T$ or
   specific internal energy $e$.
   Regions of instability in the $(\rho,e)$-plane are described
   and related to the underlying microphysical processes.}
  % results heading (mandatory)
   {Vibrational instability is found to be a common phenomenon
   at temperatures lower than the second He ionisation
   zone. The $\kappa$-mechanism is widespread under `cool'
   conditions.}
  % conclusions heading (optional), leave it empty if necessary 
   {}

   \keywords{giant planet formation --
                $\kappa$-mechanism --
                stability of gas spheres
               }

   \maketitle
%
%________________________________________________________________

\section{Data measured}
\subsection{Methods}
\subsection{Diagrams}
\subsection{Results}
\section{Data calculated - our climatic model}
\subsection{Theory}
\subsection{Results}
\section{Theory of mistakes}
\section{Biological experiments}
\subsection{Methods}
\paragraph{}
The satellite wears a insect sticky trap. The excavator took a sample of the land after landing. After the satellite was found, it was locked into an airtight metal box.The opening occurred in an airtight chamber for the biological experiments of our production.
\paragraph{}
The chamber itself works like this...(popsat funkce, následně přípravu agaru, neustálé focení, aplikaci antibiotik, přípravu vzorků, biologické omezení kultivace ; elektrické ovládání komory a etc)
\subsection{Results}
\paragraph{}
Based on the experience from previous microscopy tests, the actinomycete group of bacteria belonging to gram-positive bacteria was identified in the sample.
\paragraph{}
In total, we created four samples. But one of them was broken and we do not have its photo. It was a specimen where we covered with antibiotics the surface of the agar. Gram's method did not show any signs of life. The sample did not differ from the reference. From which it can be said that the antibiotics functioned correctly and this experiment was set up correctly. Another sample was ...
\section{References}
There we have found inspiration:
\begin{itemize}
\item\href{https://earthobservatory.nasa.gov/Features/UVB/}{NASA EO/Jeannie Allen - Ultraviolet Radiation: How it Affects Life on Earth}
\\
\item\href{https://stavba.tzb-info.cz/docu/tabulky/0000/000086_katalog.html}{TZB - Material properties catalog}
\\
\item\href{http://acmg.seas.harvard.edu/people/faculty/djj/book/bookchap7.html}{Daniel Jacob - Chapter 7 The Greeenhouse Effect in Introduction to Atmospheric Chemistry}
\\
\item\href{http://www.geocraft.com/WVFossils/greenhouse_data.html} {Geocraft - Water Vapor Rules
the Greenhouse System}
\\
\item\href{http://renewableenergy.wikia.com/wiki/Molecular_weight_of_dry_air} {Wikia - Molecular weight of dry air}
\\
\item\href{http://www.astronomynotes.com/solarsys/s3c.htm} {Nick Strobel, Astronomynotes - Surface temperature}
\end{itemize}
\section{Ilustrations}
\listoffigures








\begin{acknowledgements}
      Part of this work was supported by the German
      \emph{Deut\-sche For\-schungs\-ge\-mein\-schaft, DFG\/} project
      number Ts~17/2--1.
\end{acknowledgements}


%-------------------------------------------------------------------

\begin{thebibliography}{}

  \bibitem[1966]{baker} Baker, N. 1966,
      in Stellar Evolution,
      ed.\ R. F. Stein,\& A. G. W. Cameron
      (Plenum, New York) 333

   \bibitem[1988]{balluch} Balluch, M. 1988,
      A\&A, 200, 58

   \bibitem[1980]{cox} Cox, J. P. 1980,
      Theory of Stellar Pulsation
      (Princeton University Press, Princeton) 165

   \bibitem[1969]{cox69} Cox, A. N.,\& Stewart, J. N. 1969,
      Academia Nauk, Scientific Information 15, 1

   \bibitem[1980]{mizuno} Mizuno H. 1980,
      Prog. Theor. Phys., 64, 544
   
   \bibitem[1987]{tscharnuter} Tscharnuter W. M. 1987,
      A\&A, 188, 55
  
   \bibitem[1992]{terlevich} Terlevich, R. 1992, in ASP Conf. Ser. 31, 
      Relationships between Active Galactic Nuclei and Starburst Galaxies, 
      ed. A. V. Filippenko, 13

   \bibitem[1980a]{yorke80a} Yorke, H. W. 1980a,
      A\&A, 86, 286

   \bibitem[1997]{zheng} Zheng, W., Davidsen, A. F., Tytler, D. \& Kriss, G. A.
      1997, preprint
\end{thebibliography}

\end{document}